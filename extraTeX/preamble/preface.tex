\chapter*{Preface}

This book may be downloaded as a free PDF at \oiRedirect{textbook-openintro}{\color{black}\textbf{openintro.org}}. \vspace{3mm}

\noindent We hope readers will take away three ideas from this book in addition to forming a foundation of statistical thinking and methods.\vspace{-1mm}
\begin{enumerate}
\setlength{\itemsep}{0mm}
\item[(1)] Statistics is an applied field with a wide range of practical applications.
\item[(2)] You don't have to be a math guru to learn from real, interesting data.
\item[(3)] Data are messy, and statistical tools are imperfect. But, when you understand the strengths and weaknesses of these tools, you can use them to learn about the real~world.
\end{enumerate}


\subsection*{Textbook overview}

The chapters of this book are as follows:
\begin{description}
\setlength{\itemsep}{0mm}
\item[1. Introduction to data.] Data structures, variables, summaries, graphics, and basic data collection techniques.
\item[2. Probability (special topic).] The basic principles of probability. An understanding of this chapter is not required for the main content in Chapters~\ref{modeling}-\ref{multipleAndLogisticRegression}.
\item[3. Distributions of random variables.] Introduction to the normal model and other key distributions.
\item[4. Foundations for inference.] General ideas for statistical inference in the context of estimating the population mean.
\item[5. Inference for numerical data.] Inference for one or two sample means using the normal model and $t$ distribution, and also comparisons of many means using ANOVA.
\item[6. Inference for categorical data.] Inference for proportions using the normal and chi-square distributions, as well as simulation and randomization techniques.
\item[7. Introduction to linear regression.] An introduction to regression with two variables. Most of this chapter could be covered after Chapter~\ref{introductionToData}.
\item[8. Multiple and logistic regression.] A light introduction to multiple regression and logistic regression for an accelerated course.
\end{description}

\emph{OpenIntro Statistics} was written to allow flexibility in choosing and ordering course topics. The material is divided into two pieces: main text and special topics. The main text has been structured to bring statistical inference and modeling closer to the front of a course. Special topics, labeled in the table of contents and in section titles, may be added to a course as they arise naturally in the curriculum.


\subsection*{Videos for sections and calculators}

The \videohref[4mm]{textbook-openintro_videos} icon indicates that a section or topic has a video overview readily available. The~icons are hyperlinked in the textbook PDF, and the videos may also be found at
\begin{center}
\oiRedirect{textbook-openintro_videos}{\color{black}\textbf{www.openintro.org/stat/videos.php}}
\end{center}


\subsection*{Examples, exercises, and appendices}

Examples and Guided Practice throughout the textbook may be identified by their distinctive bullets:

\begin{example}{Large filled bullets signal the start of an example.}
Full solutions to examples are provided and may include an accompanying table or figure.
 \end{example}

\begin{exercise}
Large empty bullets signal to readers that an exercise has been inserted into the text for additional practice and guidance. Students may find it useful to fill in the bullet after understanding or successfully completing the exercise. Solutions are provided for all Guided Practice in footnotes.\footnote{Full solutions are located down here in the footnote!}
\end{exercise}

There are exercises at the end of each chapter for practice or homework assignments. Odd-numbered exercise solutions are in Appendix~\ref{eoceSolutions}. Probability tables for the normal, $t$, and chi-square distributions are in Appendix~\ref{distributionTables}.

\subsection*{OpenIntro, online resources, and getting involved}

OpenIntro is an organization focused on developing free and affordable education materials. \emph{OpenIntro Statistics} is intended for introductory statistics courses at the college level. We offer another title, \oiRedirect{textbook-books}{Advanced High School Statistics}, intended for high school courses.

We encourage anyone learning or teaching statistics to visit \oiRedirect{textbook-openintro}{\color{black}\textbf{openintro.org}} and get involved. We also provide many free online resources, including free course software. Data sets for this textbook are available on the website and through a companion R package.\footnote{Diez DM, Barr CD, \c{C}etinkaya-Rundel M. 2015. \texttt{openintro}: OpenIntro data sets and supplement functions. \oiRedirect{textbook-github_openintro}{github.com/OpenIntroOrg/openintro-r-package}.} All of these resources are free, and we want to be clear that anyone is welcome to use these online tools and resources with or without this textbook as a companion.

We value your feedback. If there is a particular component of the project you especially like or think needs improvement, we want to hear from you. You may find our contact information on the title page of this book or on the \oiRedirect{textbook-openintro_about}{About} section of \oiRedirect{textbook-openintro}{\color{black}\textbf{openintro.org}}.

\subsection*{Acknowledgements}

This project would not be possible without the dedication and volunteer hours of all those involved. No one has received any monetary compensation from this project, and we hope you will join us in extending a \emph{thank you} to all those volunteers below.

The authors would like to thank OpenIntro's staff for their involvement and ongoing contributions. We~are also very grateful to the many instructors who have provided us with valuable feedback.

